\documentclass[a4paper, parskip=half]{scrartcl}
%\usepackage{libertine}
\usepackage[english]{babel}
\usepackage[utf8]{inputenc}
\usepackage[T1]{fontenc}
\usepackage{amsmath}

\title{Proton Dynamics in Cavities}
\author{Dominik Wille}

\newcommand{\person}[1]{%
	\textsc{#1}%
}

\newcommand{\effect}[1]{%
	\textbf{#1}%
}

\newcommand{\myImage}[3]{
	\includegraphics[width = 0.8\textwidth]{img/#1}
	{\centering Figure 1: #2}
}

\begin{document}
\maketitle
\thispagestyle{empty}
\newpage
\tableofcontents
\thispagestyle{empty}
\newpage
\setcounter{page}{1}

\section{Introduction}


\section{Brownian Dynamics}
The motion of a particle such as a proton in a fluid is primarily  caused by collisions with other particles. This motion firstly was described by \person{Robert Brown} who observed the motion of minute particles of pollen in water and is widely known as \effect{brownian motion}.
\subsection{Langevin Symulation}
\subsubsection{Priciple}
The principle of a Langevin simulation generally is to consider collisions with other particles as a random force. And propergate the position of a particle over small time steps $\Delta t$.
\subsubsection{Physical background}
An approach to describe situations with brownian motion was suggested by \person{Paul Langevin} who added the random force $\mathbf{Z}(t)$ in newtons equation of motion. This stochastic term represents the collision driven force. His equation, the Langevin equation reads:

\begin{align}
m \ddot{\mathbf{r}} = -\lambda\dot{\mathbf{r}} + \mathbf{Z}(t)
\end{align}

where $m$ is the mass of the particle, $\mathbf{r}$ the position of the particle and $\lambda$ The friction constant.

Brownian dynamics can be represented with the so called \effect{overdamped langevin equation} where the $m \ddot{\mathbf{r}}$ term is neglected. The equation for the prevailing situation therefore is:

\begin{align}
\lambda\dot{\mathbf{r}} = \mathbf{Z}(t)
\end{align}

In order to get an iterable expression this expression is discretesated in time intervals $\Delta t$:

\begin{align}
\int_t^{t+ \Delta t} \dot{\mathbf{r}}(t')\, dt' &= \int_t^{t+ \Delta t} \frac{\mathbf{Z}(t')}{\lambda}\, dt' \\
\mathbf{r}(t + \Delta t) - \mathbf{r}(t) &= \frac{1}{\lambda} \int_t^{t+ \Delta t} \mathbf{Z}(t)\, dt'\\
\mathbf{r}(t + \Delta t) - \mathbf{r}(t) &\cong \boldsymbol{\zeta}(t, \varepsilon)
\end{align}

Where $\varepsilon$ is the mean step distance and $\boldsymbol{\zeta}(t, \varepsilon)$ random vector with mean norm $\varepsilon$.
Both will be discussed later on.

The position of a particle at time $t + \Delta t$ can easily derived from its position at time $t$.

\begin{align}
\mathbf{r}(t + \Delta t) = \mathbf{r}(t) + \boldsymbol{\zeta}(t, \varepsilon)
\end{align}


 
\section{Conclusion}
Bar
\end{document}