\documentclass[a4paper, parskip=half]{scrartcl}
%\usepackage{libertine}
\usepackage[english]{babel}
\usepackage[utf8]{inputenc}
\usepackage[T1]{fontenc}
\usepackage{amsmath}
\usepackage{tikz}
\usepackage{amsthm}
\usetikzlibrary{matrix}
\usepackage{amssymb}
\usepackage{xfrac}

\title{Proton Dynamics in Cavities}
\author{Dominik Wille}

\newcommand{\person}[1]{%
	\textsc{#1}%
}

\newcommand{\effect}[1]{%
	\textbf{#1}%
}

\newcommand{\myImage}[3]{
	\includegraphics[width = 0.8\textwidth]{img/#1}
	{\centering Figure 1: #2}
}

\begin{document}
\maketitle
\thispagestyle{empty}
\newpage
\tableofcontents
\thispagestyle{empty}
\newpage
\setcounter{page}{1}

\section{Introduction}


\section{Brownian Dynamics}
The motion of a particle such as a proton in a fluid is primarily  caused by collisions with other particles. This motion firstly was described by \person{Robert Brown} who observed the motion of minute particles of pollen in water and is widely known as \effect{brownian motion}.
\subsection{Langevin Symulation}
\subsubsection{Priciple}
The principle of a Langevin simulation generally is to consider collisions with other particles as a random force. And propergate the position of a particle over small time steps $\delta t$.
\subsubsection{Physical background}
An approach to describe situations with brownian motion was suggested by \person{Paul Langevin} who added the random force $\mathbf{Z}(t)$ in newtons equation of motion. This stochastic term represents the collision driven force. His equation, the Langevin equation reads:

\begin{align}
m \ddot{\mathbf{r}} = -\lambda\dot{\mathbf{r}} + \mathbf{Z}(t)
\end{align}

where $m$ is the mass of the particle, $\mathbf{r}$ the position of the particle and $\lambda$ The friction constant.

Brownian dynamics can be represented with the so called \effect{overdamped langevin equation} where the $m \ddot{\mathbf{r}}$ term is neglected. The equation for the prevailing situation therefore is:

\begin{align}
\lambda\dot{\mathbf{r}} = \mathbf{Z}(t)
\end{align}

In order to get an iterable expression this expression is discretesated in time intervals $\Delta t$:

\begin{align}
\int_t^{t+ \delta t} \dot{\mathbf{r}}(t')\, dt' &= \int_t^{t+ \Delta t} \frac{\mathbf{Z}(t')}{\lambda}\, dt' \\
\mathbf{r}(t + \delta t) - \mathbf{r}(t) &= \frac{1}{\lambda} \int_t^{t+ \Delta t} \mathbf{Z}(t)\, dt'\\
\mathbf{r}(t + \delta t) - \mathbf{r}(t) &\cong \boldsymbol{\zeta}(t, \varepsilon)
\end{align}

In the discussed 1--dimensional case $\varepsilon$ is the step size and $\boldsymbol{\zeta}(t, \varepsilon)$ will be $\varepsilon$ or $-\varepsilon$ with a chance of $50\%$ each.

The position of a particle at time $t + \Delta t$ can easily derived from its position at time $t$.

\begin{align}
\mathbf{r}(t + \delta t) = \mathbf{r}(t) + \boldsymbol{\zeta}(t, \varepsilon)
\end{align}

\subsection{Connection to diffusion}
The motion which is performed in the langevin simulation is widely known as a \effect{random walk}. The probability $p(x, n)$ that a random walk comes to a position $x$ after $n$ steps. Is given by:

\begin{align}
p(x, n) = \frac{\mathrm{Number\, of\, ways\, to\, position\,} x}{\mathrm{Total\, number\, of\, ways}} = \frac{N_x}{N}
\end{align}

Since there are 2 possible successors for every position the total number of ways  doubles every step.

\begin{align}
N = 2^n
\end{align}

The number of ways to the position $x$ after n steps can be obtained by \effect{Pascal's triangle}.

\begin{figure}[ht!]
\centering
\begin{tikzpicture}[description/.style={fill=white,inner sep=2pt}]
\matrix (m) [matrix of math nodes, row sep=1.5em,
column sep=0.3em, text height=1.5ex, text depth=0.25ex,
nodes={
        minimum width=1.0cm
    },
]
{%
\mathrm{Step/Position} &-3\varepsilon &-2\varepsilon & -1\varepsilon & 0 & \varepsilon & 2\varepsilon &3\varepsilon \\
0 & & & & 1 & & & \\
1 & & & 1 & & 1 & & \\
2 & & 1 & & 2 & & 1 &\\
3 & 1 & & 3 & & 3 & & 1\\};
\path[-] (m-2-5) edge (m-3-4)
		 (m-2-5) edge (m-3-6)
		 (m-3-4) edge (m-4-5)
		 (m-3-4) edge (m-4-3)
		 (m-3-6) edge (m-4-7)
		 (m-3-6) edge (m-4-5)
		 (m-5-6) edge (m-4-5)
		 (m-5-4) edge (m-4-3)
		 (m-5-8) edge (m-4-7)
		 (m-5-6) edge (m-4-5)
		 (m-5-6) edge (m-4-5)
		 (m-5-2) edge (m-4-3)
		 (m-5-6) edge (m-4-7)
		 (m-5-4) edge (m-4-5);
\end{tikzpicture}
\caption{Number of ways to different distances}
\end{figure}

\begin{align}
N_x &= \binom{n}{k}\\
\mathrm{with} \, \, \,\, k &= \frac{1}{2}\left(\frac{x}{\varepsilon} + n \right)
\end{align}

Therefore $p(x,n)$ follows as:

\begin{align}
p(x,n) = \binom{n}{k} \cdot 2^{-n}
\end{align}
A consequence of the \effect{de Moivre–Laplace theorem} is that for large numbers of steps $(n\rightarrow\infty)$ this expression can be approximated with the following gaussian curve:

\begin{align}
p(x,n) \cong \sqrt{\frac{2}{n \pi}} \exp\left(\frac{-x^2}{2\varepsilon^2 n} \right)
\end{align}

The probability function $p(x,n)$ has valid values only for values 
\begin{align}
x \in \{x\; |\; x = z \cdot 2 \varepsilon + \varepsilon\, (n\,\mathrm{mod}\, 2),\, z \in \mathbb{Z} \wedge |x| \leq n \varepsilon\} 
\end{align}
For other values $x$ the probability is $0$. This means that there is only one value per $2\epsilon$ interval, therefore the probability density function $\rho(x,n)$ is:
\begin{align}
\rho(x,n) = \frac{1}{2\varepsilon}\; p(x,n) = \frac{1}{\varepsilon\sqrt{2\pi n}} \exp\left(\frac{-x^2}{2\varepsilon^2 n} \right)
\end{align}
The probability density function of a \effect{random walk} fulfills the diffusion equation (\textit{Note: }$n = \sfrac{t}{\delta t}$):
\begin{align}
\frac{\partial}{\partial t} \rho(x,t) &= D \frac{\partial^2}{\partial x^2 } \rho(x,t) \\
\mathrm{with} \, \, \,\, D &= \frac{\varepsilon^2}{2 \delta t}
\end{align}
What means that a random walk can also be seen as a diffusion process.
\subsection{Mean square distance}
Obviously the expectation value after $n$ steps $\langle x_n\rangle = 0$. What can be derived from:
\begin{align}
\langle x_n\rangle = \int_{-\infty}^\infty x \cdot \rho(x,t)\; dx = 0
\end{align}
But the \effect{mean square distance} (MSD) is:
\begin{align}
\langle x_n^2\rangle = \int_{-\infty}^\infty x^2 \cdot \rho(x,t)\; dx = 2 D t
\end{align}

\end{document}